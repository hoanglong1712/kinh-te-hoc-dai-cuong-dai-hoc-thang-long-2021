\documentclass{report}
\usepackage{listings}
\usepackage[utf8]{vietnam}
\usepackage{pgfplots}
\usepackage[shortlabels]{enumitem}

\usetikzlibrary{arrows}
\usepgfplotslibrary{polar}
\usepgflibrary{shapes.geometric}
\usetikzlibrary{calc}

\pgfplotsset{compat=1.18}



\begin{document}

\setcounter{chapter}{2}
\chapter{CÁC LỰC LƯỢNG 
CUNG CẦU TRÊN THỊ 
TRƯỜNG}

\section{Điều gì xảy ra với giá và lượng cân bằng trên thị trường máy lạnh trong các
  tình huống sau:}

\begin{enumerate}[(a)]
    \item Thời tiết trở lên nóng bất thường, người bán không thay đổi lượng bán ra.
    \item  Lượng máy lạnh nhập khẩu gia tăng
    \item  Giá điện tăng cao, người bán không thay đổi lượng bán ra.
    \item  Các nhà khoa học khuyến cáo, máy lạnh có hại cho sức khỏe.
    \item  Thu nhập của người tiêu dùng giảm mạnh do suy thoái kinh tế.
    \item  Nhiều doanh nghiệp rời bỏ thị trường do chính phủ tăng thuế.
    \item  a và b xảy ra đồng thời nhưng ảnh hưởng của a mạnh hơn.
    \item  e và f xảy ra đồng thời
\end{enumerate}

\section{Cung – cầu về sản phẩm Y có dạng: $Q_S = 2P - 8$ và $Q_D = 15 - 0.5P$
  (trong đó Q tính bằng triệu tấn, P tính bằng nghìn đồng/tấn)}

\begin{enumerate}[(a)]
    \item Xác định giá và sản lượng cân bằng của sản phẩm Y.
    \\
    $Q_S = Q_D$
    \\ 
    $2P - 8 = 15 - 0.5P$
    \\
    $2.5P = 15 + 8 $
    \\
    $2.5P = 23$
    \\ 
    $P = 9.2$
    \\
    $Q_S = Q_D = 10.4$

    \begin{tikzpicture}
        \draw [->] (0,0) -- (5,0)node[right] {$Q$};
        \draw [->] (0,0) -- (0,5)node[above] {$P$};
        \draw[scale = 0.1, domain=0:20, smooth, variable=\x, color=blue] plot ({\x}, {\x /2 + 4});
        \draw[scale = 0.1, domain=0:20, smooth, variable=\x, color=red] plot ({\x}, {30 - 2 * \x});
        \filldraw[scale = 0.1, color=purple] (10.4, 9.2) circle (5pt) node[anchor=west]{A (10.4, 9.2)};
    \end{tikzpicture}

    \item Vì một lý do nào đó lượng cầu giảm 1 triệu tấn ở mọi mức giá, khi đó giá và lượng thay
          đổi như thế nào. Vẽ đồ thị minh họa câu a và câu b trên cùng một đồ thị
    \\ giá giảm , lượng cũng giảm
    \\
    cũ $Q_D = 15 - 0.5P$ 
    \\
    mới $Q_D = 14 - 0.5P$
    \\
    \begin{tikzpicture}
        \draw [->] (0,0) -- (5,0)node[right] {$Q$};
        \draw [->] (0,0) -- (0,5)node[above] {$P$};

        \draw[scale = 0.2, domain=0:20, smooth, variable=\x, color=green] plot ({\x}, {28 - 2 * \x});

        \draw[scale = 0.2, domain=0:20, smooth, variable=\x, color=blue] plot ({\x}, {\x /2 + 4});
        \draw[scale = 0.2, domain=0:20, smooth, variable=\x, color=red] plot ({\x}, {30 - 2 * \x});
        \filldraw[scale = 0.2, color=purple] (10.4, 9.2) circle (5pt) node[anchor=west]{A};
    \end{tikzpicture}
    
    \item Do giá nguyên liệu sản xuất sản phẩm Y giảm nên lượng cung tăng 10 \% tại mọi mức giá. Xác định giá và lượng cân bằng mới. Vẽ đồ thị minh họa câu a và câu c trên cùng  một đồ thị
    \\
    phương trình cũ :
    $Q_S = 2P - 8$ 

    trước đấy với số tiền 2P - 8
    chúng ta mua được $Q_S$
    do cung tăng 10 \% với mọi mức gía
    ý ở ở đây là P giữ nguyên
    \\
    thì ta sẽ mua được như sau:
    2P - 8  + 0.1 * (2P - 8)

    $Q_S = (2P - 8) + 0.1 * (2P - 8)$

    $Q_S = 2.2P - 8.8$
   
    ta tìm điểm cân băng mới

    $2.2P - 8.8 = 15 - 0.5P$

    $ 2.7P = 15 + 8.8 = 23.8 $

    $ P_{cb} = 8.81$

    $Q_{cb} = 15 - 0.5 * 8.81 =  10.59$
    
    $Q_S = 2.2P - 8.8$

    $2.2 P = Q_S + 8.8$

    $P = Q_S / 2.2 + 4$



    \begin{tikzpicture}
        \draw [->] (-3,0) -- (5,0)node[right] {$Q$};
        \draw [->] (0,0) -- (0,5)node[above] {$P$};
        \draw[scale = 0.2, domain=-10:20, smooth, variable=\x, color=green] plot ({\x}, {  \x / 2.2 + 4});
        
        \draw[scale = 0.2, domain=-10:20, smooth, variable=\x, color=blue] plot ({\x}, { 0.5 * \x  + 4});

        \filldraw[scale = 0.2, color=purple] (10.59, 8.81) circle (5pt) node[anchor=west]{A};
        
        \draw[scale = 0.2, domain=0:20, smooth, variable=\x, color=red] plot ({\x}, {30 - 2 * \x});

       
    \end{tikzpicture}

    \item Khi giá bán trên thị trường là 8 nghìn đồng/tấn thì thị trường xảy ra tình trạng gì? doanh
          thu thu được tại mức giá này là bao nhiêu?

        thiếu hụt hàng hóa 
        doanh thu tính như sau $Q_D = 15 - 0.5P$ $Q_D = 15 - 0.5 * 8$ $Q_D = 11$

        doanh thu bằng Q * P = 11 * 8 = 88

        \begin{tikzpicture}
            \draw [->] (0,0) -- (5,0)node[right] {$Q$};
            \draw [->] (0,0) -- (0,5)node[above] {$P$};
            \draw[scale = 0.2, domain=0:20, smooth, variable=\x, color=blue] plot ({\x}, {\x /2 + 4});
            \draw[scale = 0.2, domain=0:20, smooth, variable=\x, color=red] plot ({\x}, {30 - 2 * \x});
            \filldraw[scale = 0.2, color=purple] (10.4, 9.2) circle (5pt) node[anchor=west]{A};

            \draw[scale = 0.2, domain=0:20, smooth, variable=\x, color=green] (8,8) -- (11, 8);
            
            \filldraw[scale = 0.2] (8,8) circle (5pt) node[anchor=north]{B};
        \end{tikzpicture}
    \item Khi giá bán trên thị trường là 11 nghìn đồng/tấn thì thị trường xảy ra hiện tượng dư cung
          hay dư cầu? Tính mức dư cung hoặc dư cầu? Tính doanh thu thu được tại mức giá này là
          bao nhiêu?

          dư thừa hàng hóa

          doanh thu tính như sau $Q_S = 2P - 8$ $Q_S = 2 * 11 - 8$ $Q_S = 14$

           doanh thu bằng Q * P = 11 * 14 = 154
           
          \begin{tikzpicture}
            \draw [->] (0,0) -- (5,0)node[right] {$Q$};
            \draw [->] (0,0) -- (0,5)node[above] {$P$};
            \draw[scale = 0.2, domain=0:20, smooth, variable=\x, color=blue] plot ({\x}, {\x /2 + 4});
            \draw[scale = 0.2, domain=0:20, smooth, variable=\x, color=red] plot ({\x}, {30 - 2 * \x});
            \filldraw[scale = 0.2, color=purple] (10.4, 9.2) circle (5pt) node[anchor=west]{A};

            \draw[scale = 0.2, domain=0:20, smooth, variable=\x, color=green] (14,11) -- (9.5, 11);

            \filldraw[scale = 0.2] (9.5, 11) circle (5pt) node[anchor=east]{B};

        \end{tikzpicture}
\end{enumerate}



\section{ Cho số liệu về cung – cầu sản phẩm A như sau:}
\begin{tabular}{|c|c|c|}
    Giá (100đ/ 1kg) & Lượng cầu (kg) & Lượng cung(kg) \\
    \hline
    7               & 20             & 11             \\
    \hline
    8               & 19             & 13             \\
    \hline
    9               & 18             & 15             \\
    \hline
\end{tabular}


\begin{enumerate}[(a)]
    \item Viết phương trình đường cung, đường cầu, xác định giá và lượng cân bằng. Doanh thu
          tại trạng thái cân bằng.
          \\
          chúng ta nhắc lại về phương pháp tính phương trình đường thẳng trong hệ tọa độ Đề Các

          phương trình đường thẳng
          đi qua 2 điểm trong hệ tọa độ Đề Các

          ta có trục Ox và trục Oy

          $A * (x - x_0) + B * (y - y_0) = 0$

          ta đã có $x_0$ và $y_0$
          vi dụ $x_0 = 20$ và $y_0 = 7$

          ta cần tìm A và B
          chúng ta nhơ lại rằng (A, B) là véc tơ pháp tuyến  của đương thẳng đi qua 2 điểm cho trước

          muốn tìm  vec tơ pháp tuyến ta cần tìm véc tơ chỉ phương

          vec tơ chỉ phương sẽ tính như sau

          giả sử chúng tâ có 2 điểm
          $M(20, 7)$ $N(19, 8)$

          véc tơ MN = (19 - 20, 8 - 7) = (-1, 1)

          vậy ta đã có véc tơ chỉ phương

          vec tơ pháp tuyến tính như sau

          công thức
          chỉ phương = (C, D)
          pháp tuyến = (-D, C)

          MN =  (-1, 1)
          $\Rightarrow$ pháp tuyến = (-1, -1)

          phương trình đường cầu

          $A * (x - x_0) + B * (y - y_0) = 0$

          (A, B) = (-1-, -1)

          $x_0 = 20$ $y_0= 7$

          $-1 * (x - 20) + (-1) * (y - 7) = 0$

          $-x + 20 - y + 7 = 0$

          $-x - y + 27 = 0$

          $x = 27 - y$

          $Q_D = 27 - P$

          đường cung
          MN = (13 - 11, 8 - 7) = (2, 1)
          pháp tuyến = (-1, 2)

          phương trình đường cung

          $-1 * (x - 11) + (2) * (y - 7) = 0$

          $-x + 11 + 2y - 7 = 0$

          $-x + 2y + 4 = 0$

          $4 + 2y = x$

          $Q_S = 4 + 2P$



          kết luận
          ta có

          $Q_S = 2P + 4$,
          $Q_D = 27 - P$

          $P = Q_S / 2 - 2$ $y = x / 2 - 2$

          $P = 27 - Q_D$

          tính điểm giao của 2 đường thăng - điểm cân bằng

          $Q_S = Q_D$

          $2P + 4 = 27 - P$

          $3P = 23 \rightarrow P = 7.6, Q = 19.4$

          Doanh thu tại trạng thái cân bằng.:
          P * Q = 7.6 * 19.4

          \begin{tikzpicture}
              \draw [->] (0,0) -- (5,0)node[right] {$Q$};
              \draw [->] (0,0) -- (0,5)node[above] {$P$};
              \draw[scale = 0.1, domain=0:30, smooth, variable=\x, color=blue] plot ({\x}, {\x /2 - 2});
              \draw[scale = 0.1, domain=0:30, smooth, variable=\x, color=red] plot ({\x}, {27 - \x});
              \filldraw[scale = 0.1, color=purple] (19.4, 7.6) circle (5pt) node[anchor=west]{A};
          \end{tikzpicture}


    \item Vì lý do nào đó, lượng cung sản phẩm A tăng lên một lượng là 6 kg ở mỗi mức giá. Hãy
          xác định mức giá và sản lượng, tổng doanh thu tại trạng thái cân bằng mới?.

          cũ
          $Q_S = 2P + 4$,
          $Q_D = 27 - P$

          mới
          $Q_S = (2P + 4) + 6$,
          $Q_D = 27 - P$

          $ \Rightarrow Q_S = 2P + 10$,
          $Q_D = 27 - P$

          $Q_S = Q_D$

          $2P + 10 = 27 - P$

          $3P = 17 \rightarrow P = 6.3, Q = 21.7 \rightarrow$ tổng doanh thu là 6.3 * 21.7

    \item Giả sử Chính phủ áp đặt giá bán trên thị trường là 11 nghìn đồng/kg và hứa mua hết phần
          sản phẩm thừa, thì số tiền chính phủ phải chi ra là bao nhiêu?

    đường thẳng song song với trục hoành
    y = 11 là đường giá cố định của chính phủ

    ta cần tính giáo của đường cung với đường áp giá để tìm ra lượng hàng cần tiêu thụ

    $Q_S = 2P + 4$,
    thay P  = 11 vào ta có 

    $Q_S = 2 * 11 + 4 = 26$,

    $Q_D = 27 - P$
    thay P  = 11 vào ta có 

    $Q_D = 27 - 11 = 16$

    lượng Dư thừa = $Q_S - Q_D = 26 - 16 = 10$

    vậy chính phủ cần mua 10 kg
    , số tiền bỏ ra là 10 * 11 = 110 

          \begin{tikzpicture}
            \draw [->] (0,0) -- (5,0)node[right] {$Q$};
            \draw [->] (0,0) -- (0,5)node[above] {$P$};
            \draw[scale = 0.1, domain=0:30, smooth, variable=\x, color=blue] plot ({\x}, {\x /2 - 2});
            \draw[scale = 0.1, domain=0:30, smooth, variable=\x, color=red] plot ({\x}, {27 - \x});
            \filldraw[scale = 0.1, color=purple] (19.4, 7.6) circle (5pt) node[anchor=west]{A};

            \draw[scale = 0.1, domain=0:30, smooth, variable=\y, color=green] plot ({\y}, {11});

            \filldraw[scale = 0.1, color=purple] (26, 11) circle (5pt) node[anchor=south]{S};

            \filldraw[scale = 0.1, color=purple] (16, 11) circle (5pt) node[anchor=south]{D};
        \end{tikzpicture}
\end{enumerate}

\begin{tikzpicture}
    \draw [->] (0,0) -- (5,0)node[right] {$Q$};
    \draw [->] (0,0) -- (0,5)node[above] {$P$};
    \draw[scale = 0.1, domain=0:20, smooth, variable=\x, color=blue] plot ({\x}, {\x /2 + 4});
    \draw[scale = 0.1, domain=0:20, smooth, variable=\x, color=red] plot ({\x}, {30 - 2 * \x});
    \filldraw[scale = 0.1, color=black] (10.4, 9.2) circle (5pt) node[anchor=west]{A};
\end{tikzpicture}

\end{document}