\documentclass{report}
\usepackage{listings}
\usepackage[utf8]{vietnam}
\usepackage{pgfplots}
\usepackage[shortlabels]{enumitem}

\usetikzlibrary{arrows}
\usepgfplotslibrary{polar}
\usepgflibrary{shapes.geometric}
\usetikzlibrary{calc}

\pgfplotsset{compat=1.18}



\begin{document}

\setcounter{chapter}{2}
\chapter{CÁC LỰC LƯỢNG 
CUNG CẦU TRÊN THỊ 
TRƯỜNG}

\section{Điều gì xảy ra với giá và lượng cân bằng trên thị trường máy lạnh trong các
  tình huống sau:}

\begin{enumerate}[(a)]
    \item Thời tiết trở lên nóng bất thường, người bán không thay đổi lượng bán ra.
    \item  Lượng máy lạnh nhập khẩu gia tăng
    \item  Giá điện tăng cao, người bán không thay đổi lượng bán ra.
    \item  Các nhà khoa học khuyến cáo, máy lạnh có hại cho sức khỏe.
    \item  Thu nhập của người tiêu dùng giảm mạnh do suy thoái kinh tế.
    \item  Nhiều doanh nghiệp rời bỏ thị trường do chính phủ tăng thuế.
    \item  a và b xảy ra đồng thời nhưng ảnh hưởng của a mạnh hơn.
    \item  e và f xảy ra đồng thời
\end{enumerate}

\section{Cung – cầu về sản phẩm Y có dạng: $Q_S = 2P - 8$ và $Q_D = 15 - 0.5P$
  (trong đó Q tính bằng triệu tấn, P tính bằng nghìn đồng/tấn)}

\begin{enumerate}[(a)]
    \item Xác định giá và sản lượng cân bằng của sản phẩm Y.
    \\
    $Q_S = Q_D$
    \\ 
    $2P - 8 = 15 - 0.5P$
    \\
    $2.5P = 15 + 8 $
    \\
    $2.5P = 23$
    \\ 
    $P = 9.2$
    \\
    $Q_S = Q_D = 10.4$

    \begin{tikzpicture}
        \draw [->] (0,0) -- (5,0)node[right] {$Q$};
        \draw [->] (0,0) -- (0,5)node[above] {$P$};
        \draw[scale = 0.1, domain=0:20, smooth, variable=\x, color=blue] plot ({\x}, {\x /2 + 4});
        \draw[scale = 0.1, domain=0:20, smooth, variable=\x, color=red] plot ({\x}, {30 - 2 * \x});
        \filldraw[scale = 0.1, color=purple] (10.4, 9.2) circle (5pt) node[anchor=west]{A (10.4, 9.2)};
    \end{tikzpicture}

    \item Vì một lý do nào đó lượng cầu giảm 1 triệu tấn ở mọi mức giá, khi đó giá và lượng thay
          đổi như thế nào. Vẽ đồ thị minh họa câu a và câu b trên cùng một đồ thị
    \\ giá giảm , lượng cũng giảm
    \\
    cũ $Q_D = 15 - 0.5P$ 
    \\
    mới $Q_D = 14 - 0.5P$
    \\
    \begin{tikzpicture}
        \draw [->] (0,0) -- (5,0)node[right] {$Q$};
        \draw [->] (0,0) -- (0,5)node[above] {$P$};

        \draw[scale = 0.2, domain=0:20, smooth, variable=\x, color=green] plot ({\x}, {28 - 2 * \x});

        \draw[scale = 0.2, domain=0:20, smooth, variable=\x, color=blue] plot ({\x}, {\x /2 + 4});
        \draw[scale = 0.2, domain=0:20, smooth, variable=\x, color=red] plot ({\x}, {30 - 2 * \x});
        \filldraw[scale = 0.2, color=purple] (10.4, 9.2) circle (5pt) node[anchor=west]{A};
    \end{tikzpicture}
    
    \item Do giá nguyên liệu sản xuất sản phẩm Y giảm nên lượng cung tăng 10 \% tại mọi mức giá. Xác định giá và lượng cân bằng mới. Vẽ đồ thị minh họa câu a và câu c trên cùng  một đồ thị
    \\
    phương trình cũ :
    $Q_S = 2P - 8$ 

    trước đấy với số tiền 2P - 8
    chúng ta mua được $Q_S$
    do cung tăng 10 \% với mọi mức gía
    ý ở ở đây là P giữ nguyên
    \\
    thì ta sẽ mua được như sau:
    2P - 8  + 0.1 * (2P - 8)

    $Q_S = (2P - 8) + 0.1 * (2P - 8)$

    $Q_S = 2.2P - 8.8$
   
    ta tìm điểm cân băng mới

    $2.2P - 8.8 = 15 - 0.5P$

    $ 2.7P = 15 + 8.8 = 23.8 $

    $ P_{cb} = 8.81$

    $Q_{cb} = 15 - 0.5 * 8.81 =  10.59$
    
    $Q_S = 2.2P - 8.8$

    $2.2 P = Q_S + 8.8$

    $P = Q_S / 2.2 + 4$



    \begin{tikzpicture}
        \draw [->] (-3,0) -- (5,0)node[right] {$Q$};
        \draw [->] (0,0) -- (0,5)node[above] {$P$};
        \draw[scale = 0.2, domain=-10:20, smooth, variable=\x, color=green] plot ({\x}, {  \x / 2.2 + 4});
        
        \draw[scale = 0.2, domain=-10:20, smooth, variable=\x, color=blue] plot ({\x}, { 0.5 * \x  + 4});

        \filldraw[scale = 0.2, color=purple] (10.59, 8.81) circle (5pt) node[anchor=west]{A};
        
        \draw[scale = 0.2, domain=0:20, smooth, variable=\x, color=red] plot ({\x}, {30 - 2 * \x});

       
    \end{tikzpicture}

    \item Khi giá bán trên thị trường là 8 nghìn đồng/tấn thì thị trường xảy ra tình trạng gì? doanh
          thu thu được tại mức giá này là bao nhiêu?

        thiếu hụt hàng hóa 
        doanh thu tính như sau $Q_D = 15 - 0.5P$ $Q_D = 15 - 0.5 * 8$ $Q_D = 11$

        doanh thu bằng Q * P = 11 * 8 = 88

        \begin{tikzpicture}
            \draw [->] (0,0) -- (5,0)node[right] {$Q$};
            \draw [->] (0,0) -- (0,5)node[above] {$P$};
            \draw[scale = 0.2, domain=0:20, smooth, variable=\x, color=blue] plot ({\x}, {\x /2 + 4});
            \draw[scale = 0.2, domain=0:20, smooth, variable=\x, color=red] plot ({\x}, {30 - 2 * \x});
            \filldraw[scale = 0.2, color=purple] (10.4, 9.2) circle (5pt) node[anchor=west]{A};

            \draw[scale = 0.2, domain=0:20, smooth, variable=\x, color=green] (8,8) -- (11, 8);
            
            \filldraw[scale = 0.2] (8,8) circle (5pt) node[anchor=north]{B};
        \end{tikzpicture}
    \item Khi giá bán trên thị trường là 11 nghìn đồng/tấn thì thị trường xảy ra hiện tượng dư cung
          hay dư cầu? Tính mức dư cung hoặc dư cầu? Tính doanh thu thu được tại mức giá này là
          bao nhiêu?

          dư thừa hàng hóa

          doanh thu tính như sau $Q_S = 2P - 8$ $Q_S = 2 * 11 - 8$ $Q_S = 14$

           doanh thu bằng Q * P = 11 * 14 = 154
           
          \begin{tikzpicture}
            \draw [->] (0,0) -- (5,0)node[right] {$Q$};
            \draw [->] (0,0) -- (0,5)node[above] {$P$};
            \draw[scale = 0.2, domain=0:20, smooth, variable=\x, color=blue] plot ({\x}, {\x /2 + 4});
            \draw[scale = 0.2, domain=0:20, smooth, variable=\x, color=red] plot ({\x}, {30 - 2 * \x});
            \filldraw[scale = 0.2, color=purple] (10.4, 9.2) circle (5pt) node[anchor=west]{A};

            \draw[scale = 0.2, domain=0:20, smooth, variable=\x, color=green] (14,11) -- (9.5, 11);

            \filldraw[scale = 0.2] (9.5, 11) circle (5pt) node[anchor=east]{B};

        \end{tikzpicture}
\end{enumerate}

\begin{tikzpicture}
    \draw [->] (0,0) -- (5,0);
    \draw [->] (0,0) -- (0,5);
    \draw [red]  (2,3) -- (0,4);
\end{tikzpicture}

\end{document}