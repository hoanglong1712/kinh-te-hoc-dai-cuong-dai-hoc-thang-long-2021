\chapter{HỆ SỐ CO GIÃN}

%\setcounter{section}{5}
\section{Tính hệ số co giãn của cầu theo giá của các hàng hóa thịt bò, áo sơ mi, biết rằng:}

\begin{enumerate}[a.]
  \item  Giá thịt bò ban đầu là 1,7 \$/kg thì bán được 116.250 kg. Khi hạ giá 0,2\$ thì lượng bán tăng thêm 7.500kg.

        ta có công thức như sau
        \[ E_{DP} = \frac{\%\Delta Q_D}{\%\Delta P} \]
        với \\
        $\%\Delta Q_D$: phần trăm thay đổi của lượng cầu \\
        $\%\Delta P$: phần trăm thay đổi của giá

        ta có \\
        giá cũ là 1.7, giá hạ 0.2 tức là hạ $0.2/1.7 * 100 = 11.76\%$\\
        cầu cũ là 116.25 lượng cầu tăng 7.5 tức là tăng
        $7.5/116.25 * 100 = 6.45\%$

        vậy $E_{DP} = 6.45/(-11.76) = -0.548$


  \item Áo sơ mi giá ban đầu 8,1\$/chiếc thì bán được 19.500 chiếc. Khi tăng giá 0,2\$ thì lượng bán giảm 5000 chiếc

        ta có \\
        giá cũ là 8.1, giá tăng 0.2 tức là tăng $0.2/8.1 * 100 = 2.46\%$\\
        cầu cũ là 19500 lượng cầu giảm 5000 tức là giảm
        $5000/19500 * 100 = 25.64\%$

        vậy $E_{DP} = -25.64/2.46 = -10.42$
\end{enumerate}

\section{Hàm cầu về bánh mỳ của công ty Kinh Đô như sau: $Q_D = 40 - 5P$ (Q :nghìn chiếc ; P: nghìn đồng/chiếc)}

\begin{enumerate}[a.]
  \item  Tính hệ số co giãn của cầu tại mức giá bẳng 3; và khi giá tăng từ 2 lên 5 theo phương
        pháp trung điểm.

        nhắc lại về phương pháp trung điểm

        cho khoảng giá
        \[ E_{DP} = \frac{\%\Delta Q_D}{\%\Delta P} =
          \frac{\frac{\Delta Q_D}{Q_D} \times 100 \% }{ \frac{\Delta P}{P} \times 100 \%  } =
          \frac{\Delta Q_D}{\Delta P}  \times
          \frac{\frac{P_1 + P_2}{2}}{\frac{Q_{D_1} + Q_{D_2}}{2}} \]

        cho điểm
        \[ E_{DP} = \frac{\%\Delta Q_D}{\%\Delta P} =
          \frac{\Delta Q_D}{\Delta P} \times \frac{P}{Q_D} = Q_D' \times \frac{P}{Q_D} \]
        hoặc
        \[ E_{DP} = \frac{\%\Delta Q_D}{\%\Delta P} =
          \frac{\Delta Q_D}{\Delta P} \times \frac{P}{Q_D} = \frac{1}{P_D'} \times \frac{P}{Q_D} \]

        lưu ý, $Q_D'$ là đạo hàm của $Q_D$ theo P còn $P_D'$ là đạo hàm của $P_D$ theo Q, nếu ai đó quên thì có thể giở sách giáo khoa lớp 11 môn giải tích để đọc lại \url{https://drive.google.com/file/d/1Ygj-Lw40zs6JHfA--VnH\_Bs-gk2X\_aek/view}

        cụ thể chúng ta sẽ làm như sau

        Tính hệ số co giãn của cầu tại mức giá bẳng 3;
        ta áp dụng công thức
        \[ E_{DP} = \frac{\%\Delta Q_D}{\%\Delta P} =
          \frac{\Delta Q_D}{\Delta P} \times \frac{P}{Q_D} = Q_D' \times \frac{P}{Q_D} \]

        $Q_D' = ( 40 - 5P)' = -5$
        do đạo hàm của 40 = 0 và đạo hàm của -5P = -5 * đạo hàm của P = (-5) * 1 = -5
        \[ E_{DP} = \frac{\%\Delta Q_D}{\%\Delta P} = -5 \times \frac{3}{40 - 5 * 3}
          = -5 \times \frac{3}{25} = -0.6 \]

        và khi giá tăng từ 2 lên 5 theo phương  pháp trung điểm.

        \[ E_{DP} = \frac{\%\Delta Q_D}{\%\Delta P} =
          \frac{\Delta Q_D}{\Delta P}  \times
          \frac{\frac{P_1 + P_2}{2}}{\frac{Q_{D_1} + Q_{D_2}}{2}}
        \]
        nhắc lại phương trình cầu $Q_D = 40 - 5P$

        $Q_2 = 40 - 5 * 2 = 30$

        $Q_5 = 40 - 5 * 5 = 15$

        $\Delta Q_D = Q_2 - Q_5 = 30 - 15 = 15$

        $\Delta P = 2 - 5 = -3$

        \[ E_{DP} = \frac{\%\Delta Q_D}{\%\Delta P} =
          \frac{15}{-3}  \times
          \frac{\frac{2 + 5}{2}}{\frac{30 + 15}{2}} =
          -5  \times
          \frac{7}{45} = \frac{-7}{9} = -0.777
        \]



  \item Để tăng tổng doanh thu công ty nên áp dụng chính sách giá nào nếu hiện tại công ty
        đang bán ở mức giá P = 3 và P = 5? Giải thích tại sao?

        chúng ta có bảng sau

        \begin{tabular}{ |c|c|c| }
          \hline
                         & \textbf{Khi tăng P}               & \textbf{Khi giảm P}                 \\
          \hline
          $|E_{DP}| < 1$ & \% tăng lên của P luôn lớn        & \% giảm xuống của P luôn            \\
                         & hơn \% giảm xuống của $Q_D$       & lớn hơn \% tăng lên của $Q_D$       \\
                         & $\Rightarrow$ P tăng thì TR tăng  & $\Rightarrow$ P giảm thì TR giảm    \\
          \hline
          $|E_{DP}| > 1$ & \% tăng lên của P luôn nhỏ        & \% giảm xuống của P luôn            \\
                         & hơn \% giảm xuống của $Q_D$       & nhỏ hơn \% tăng lên của $Q_D$       \\
                         & $\Rightarrow$ P tăng thì TR giảm  & $\Rightarrow$ P giảm thì TR tăng    \\
          \hline
          $|E_{DP}| = 1$ & \% giảm xuống của $Q_D$ bằng      & \% tăng lên của $Q_D$ bằng          \\
                         & đúng với \% tăng lên của P        & đúng với \% giảm xuống của          \\
                         & $\Rightarrow$ P tăng TR không đổi & P $\Rightarrow$ P giảm TR không đổi \\
          \hline
        \end{tabular}


        tại mức giá bẳng 3;
        ta áp dụng công thức
        \[ E_{DP} = \frac{\%\Delta Q_D}{\%\Delta P} =
          \frac{\Delta Q_D}{\Delta P} \times \frac{P}{Q_D} = Q_D' \times \frac{P}{Q_D} \]

        $Q_D' = ( 40 - 5P)' = -5$
        do đạo hàm của 40 = 0 và đạo hàm của -5P = -5 * đạo hàm của P = (-5) * 1 = -5
        \[ E_{DP_3} = \frac{\%\Delta Q_D}{\%\Delta P} = -5 \times \frac{3}{40 - 5 * 3}
          = -5 \times \frac{3}{25} = -0.6 \]

        $|E_{DP_3}| < 1$ nên chúng ta sẽ tăng P để tăng doanh thu

        tại mức giá bẳng 5;
        \[ E_{DP_5} = \frac{\%\Delta Q_D}{\%\Delta P} = -5 \times \frac{5}{40 - 5 * 5}
          = -5 \times \frac{5}{15} = -1.6 \]

        $|E_{DP_5}| > 1$ nên chúng ta sẽ giảm P để tăng doanh thu




  \item Tổng doanh thu của công ty lớn nhất ở mức giá nào?

        chúng ta cần tìm giá trị lớn nhất của $TR = Q * P = (40 - 5P) * P = 40P - 5P^2$

        \begin{tikzpicture}
          \draw [->] (0,0) -- (5,0)node[right] {$P$};
          \draw [->] (0,0) -- (0,5)node[above] {$TR$};

          \draw[scale = 0.05, domain=0:8, smooth, variable=\x, color=blue] plot ({\x}, {40 * \x -
              5 * \x * \x});

          \filldraw[scale = 0.05, color=red] (4, 80) circle (25pt) node[anchor=west]{MAX};
        \end{tikzpicture}

        ta có thể thấy đồ thị có điểm cực đại, và chúng ta cần tìm điểm cực đại đó
        mọi người có thể xem lại sách giải tích lớp 12

        đầu tiên tính đạo hàm TR' = 40 - 10P \\
        TR' = 0 khi 10P = 40 hay P = 4 \\
        tại đó $Q_D = 20$ vậy TR = 20 * 4 = 80



\end{enumerate}

\section{ Giả sử thu nhập hàng tháng của hộ gia đình giảm từ \$10.000 xuống còn \$6.000,
  trong khi tiêu dùng hàng tháng về sản phẩm X của họ tăng từ 200 lên 400}

\begin{enumerate}[a.]
  \item Hãy tính hệ số co giãn của cầu theo thu nhập đối với hàng hóa X.

        chúng ta nhắc lại công thức tính hệ số co giãn của cầu theo thu nhập

        \[ E_{DI} =
          \frac{\% \Delta Q_D}{\% \Delta I}
          = \frac{\frac{\Delta Q_D}{Q_D} * 100 \% }{\frac{\Delta I}{I} * 100 \% }
          = \frac{\Delta Q_D}{\Delta I} * \frac{I}{Q_D}
        \]
        trong đó

        $\% \Delta Q_D$ là phần trăm thay đổi của lượng cầu

        $\% \Delta I$ là phần trăm thay đổi của thu nhập

        $ I = \frac{I_1 + I_2}{2}$

        $ Q_D = \frac{Q_{D_1} + Q_{D_2}}{2}$


        ta sẽ tính như sau

        $ I = \frac{10.000 + 6.000}{2} = 8.000$

        $ Q_D = \frac{200 + 400}{2} = 300$

        $\Delta Q_D = 200 - 400 = -200$

        $\Delta I = 10.000 - 6.000 = 4.000$

        kết quả như sau

        \[ E_{DI} = \frac{-200}{4000} * \frac{-8000}{300} = \frac{-4}{3}  \]


  \item X là hàng hóa thông thường hay hàng hóa thứ cấp? Giải thích.

        ta $E_{DI} = \frac{-4}{3}$ đó là 1 số âm nên nó có I và $Q_D$ vận động ngược chiều
        nên nó là hàng hóa thứ cấp

        định nghĩa về hàng hóa thứ cấp có trong đường dẫn sau
        \url{https://dragonlend.vn/dragonlend-blog/phan-biet-hang-hoa-thong-thuong-va-hang-hoa-thu-cap/}

\end{enumerate}

\section{Hàm cầu của hàng hóa A theo thu nhập được biểu diễn như sau: $Q = 100I + 1000$ }

\begin{enumerate}[a.]
  \item Hàng A là hàng hóa thông thường hay thứ cấp?

        để xác định xem A là hàng thông thường hay thứ cấp
        chúng ta cần xác định xem $E_{DI}$ của nó là âm hay dương

        ta có công thức tính $E_{DI}$ theo điểm như sau
        \[ E_{DI} =
          \frac{\% \Delta Q_D}{\% \Delta I}
          = \frac{\frac{\Delta Q_D}{Q_D} * 100 \% }{\frac{\Delta I}{I} * 100 \% }
          = Q_D' * \frac{I}{Q_D}
        \]

        chúng ta lưu ý rằng I và $Q_D$ thì luôn là số dương vì trong kinh tế người ta chắc hẳn không quan tâm đến giá âm và lượng hàng âm

        vậy dấu của $E_{DI}$ phụ thuộc vào dấu của $Q_D'$

        như đã nói ở bài trước $Q_D'$ là đạo hàm của $Q_D$ theo P
        các bạn nào chưa nhớ ra đạo hàm là gì thì có thể giở sách giáo khoa toán giải tích lớp 11 để xem lại
        \url{https://www.o-study.net/}

        với phương trình đã cho $Q = 100I + 1000$ ta có $Q' = 100$
        đó là 1 số dương, vậy theo định nghĩa trong slide tuần 4 trang 8, đây là hàng hóa thông thường


  \item Tính $E_{DI}$ tại mức thu nhập là 10.

        tại I = 10 ta có Q = 100 * 10 + 1000 = 2000

        áp dụng công thức tính $E_{DI}$  tại điểm ta có

        \[ E_{DI} =
          Q_D' * \frac{I}{Q_D}
          = 100 * \frac{10}{2000}
          = 0.5
        \]

  \item  Khi thu nhập tăng từ 10 lên 20 thì hệ số co giãn của cầu theo thu nhập là bao nhiêu?

        ở đây phải tính hệ số co giãn theo khoảng
        và chúng ta có công thức sau

        \[ E_{DI} =
          \frac{\% \Delta Q_D}{\% \Delta I}
          = \frac{\frac{\Delta Q_D}{Q_D} * 100 \% }{\frac{\Delta I}{I} * 100 \% }
          = \frac{\Delta Q_D}{\Delta I} * \frac{I}{Q_D}
        \]

        $ I = \frac{10 + 20}{2} = 15$

        $ Q_D = \frac{100 * 10 + 1000 + 100 * 20 + 1000}{2} = 3000$

        $\Delta Q_D = 100 * 10 + 1000 - (100 * 20 + 1000) = -1000$

        $\Delta I = 10 - 20 = -10$

        kết quả như sau

        \[ E_{DI} = \frac{-1000}{-10} * \frac{15}{3000} = 0.5  \]

\end{enumerate}

\section{ Lượng cầu về cam khi giá quýt thay đổi được cho ở biểu sau:}

\begin{tabular}{|c|c|}
  \hline
  P quít (nghìn đồng / kg) & Q cam (tấn) \\
  \hline
  5                        & 20          \\
  \hline
  6                        & 23          \\
  \hline
  7                        & 25          \\
  \hline
  8                        & 28          \\
  \hline
  9                        & 30          \\
  \hline
\end{tabular}

\begin{enumerate}[a.]
  \item Tính hệ số co giãn chéo giữa cầu về cam và quýt khi giá quýt thay đổi từ 5 lên 6 nghìn
        đồng/kg? từ 6 lên 8 nghìn đồng/kg.

        chúng ta sử dụng công thức tính hệ số co giãn chéo trên khoảng

        \[ E_{DC} = \frac{\% \Delta Q_{DX}}{\% \Delta P_{Y}}
          =  \frac{ \Delta Q_{DX}}{\Delta P_{Y}} * \frac{P_{Y}}{Q_{DX}} \]

        với

        $P_{Y} = \frac{P_{Y_1} + P_{Y_2}}{2}$

        $Q_{DX} = \frac{Q_{DX_1} + Q_{DX_2}}{2}$

        từ 5 lên 6 nghìn
        thay vào công thức tính như sau

        $P_{Y} = \frac{5 +6 }{2}$

        $Q_{DX} = \frac{20 + 23}{2}$

        $\Delta Q_{DX} = 20 -23 = -3$

        $\Delta P_{Y} = 5 -6 = -1$

        \[ E_{DC} = \frac{ \Delta Q_{DX}}{\Delta P_{Y}} * \frac{P_{Y}}{Q_{DX}}
          = \frac{-3}{-1} * \frac{11}{33} = 1 \]


        từ 6 lên 8 nghìn
        thay vào công thức tính như sau

        $P_{Y} = \frac{6 + 8 }{2}$

        $Q_{DX} = \frac{23 + 28}{2}$

        $\Delta Q_{DX} = 23 -28 = -5$

        $\Delta P_{Y} = 6 - 8 = -3$

        \[ E_{DC} = \frac{ \Delta Q_{DX}}{\Delta P_{Y}} * \frac{P_{Y}}{Q_{DX}}
          = \frac{-5}{-3} * \frac{14}{51} = \frac{70}{153} \]


  \item Mối quan hệ giữa cam và quýt

        ta thấy $E_{DC}$ dương do đó X và Y hoạt động cùng chiều và là hai mặt hàng thay thế
\end{enumerate}

\section{ Một công ty ước lượng được hàm cầu đối với sản phẩm của mình như sau:
  $Q_X = 1000 - 0.6P_Y$ . Trong đó $Q_X$ là lượng cầu đối với hàng hóa X do công ty kinh doanh và $P_Y$ là giá của hàng hóa Y có liên quan với hàng hóa X}


\begin{enumerate}[a.]
  \item Xác định mối quan hệ giữa 2 hàng hóa X và Y?

        ở đây đầu để bài chỉ cung cấp phương trình nên hệ số co giãn sẽ tính theo công thức
        hệ số co giãn theo điểm

        \[ E_{DC} = \frac{\% \Delta Q_{DX}}{\% \Delta P_{Y}}
          =  \frac{ \Delta Q_{DX}}{\Delta P_{Y}} * \frac{P_{Y}}{Q_{DX}}
          = Q_D' *  \frac{P_{Y}}{Q_{DX}} \]

        như đã nói trong bài trước, giá trị của $E_{DC}$ âm hay dương đều phụ thuộc vào giá trị của $Q_D'$ là âm hay dương

        $Q_D'$ là đạo hàm của $Q_{DX}$ theo $P_Y$  chi tiết cách tính đạo hàm , mọi người có thể tìm hiểu lại sách giáo khoa toán giải tích lớp 11 tại đường link sau \url{https://www.o-study.net/} hoặc xem lại 1 vài video trước video này, mình đã hướng dẫn chi tiết về đạo hàm dùng trong trường hợp bài toán kinh tế này

        ta có $Q_D' = (1000 - 0.6P_Y)' = -0.6$

        do đó $E_{DC}$ âm , vậy X và Y là hai hàng hóa bổ sung cho nhau theo trang 9 của slide tuần 4

  \item Tính hệ số co giãn chéo của cầu hàng hóa X tại mức giá của hàng hóa Y là 40.

        áp dụng công thức hệ số co giãn theo điểm

        \[ E_{DC} = \frac{\% \Delta Q_{DX}}{\% \Delta P_{Y}}
          =  \frac{ \Delta Q_{DX}}{\Delta P_{Y}} * \frac{P_{Y}}{Q_{DX}}
          = Q_D' *  \frac{P_{Y}}{Q_{DX}} \]

        ta có

        \[ E_{DC} = Q_D' *  \frac{P_{Y}}{Q_{DX}}
          = -0.6 *  \frac{40}{1000 - 0.6 * 40}
          = -0.6 * \frac{40}{976} = -0.245 \]

  \item Hãy xác định hệ số co giãn chéo của cầu hàng hóa X khi giá hàng hóa Y thay đổi
        trong khoảng từ 80 đến 100

        áp dụng công thức
        \[ E_{DC} = \frac{\% \Delta Q_{DX}}{\% \Delta P_{Y}}
          =  \frac{ \Delta Q_{DX}}{\Delta P_{Y}} * \frac{P_{Y}}{Q_{DX}} \]

        với

        $P_{Y} = \frac{P_{Y_1} + P_{Y_2}}{2}$

        $Q_{DX} = \frac{Q_{DX_1} + Q_{DX_2}}{2}$

        từ 5 lên 6 nghìn
        thay vào công thức tính như sau

        $P_{Y} = \frac{80 + 100 }{2}$

        $Q_{DX} = \frac{1000 - 0.6 * 80 + 1000 - 0.6 * 100}{2}$

        $\Delta Q_{DX} = 1000 - 0.6 * 80 - 1000 + 0.6 * 100 = 12$

        $\Delta P_{Y} = 80 - 100 = -20$

        \[ E_{DC} = \frac{ \Delta Q_{DX}}{\Delta P_{Y}} * \frac{P_{Y}}{Q_{DX}}
          = \frac{-20}{12} * \frac{180}{1892} = \frac{-5}{3} * \frac{180}{1892} = -0.158\]


\end{enumerate}

\section{ Một người tiêu dùng, tháng nào cũng mua hai sản phẩm X và Y, thu nhập sẵn
  có của ông ta thay đổi qua các tháng. Chúng ta có 6 quan sát những lượng sản phẩm
  X được tiêu thụ trong khi giá của X, giá của Y và thu nhập sẵn có thay đổi như sau:}


\begin{tabular}{|c|c|c|c|c|}
  \hline
  Quan sát & Lượng cầu X & Giá của X & Giá của Y & Thu nhập sẵn có \\
  \hline
  1        & 20          & 10        & 15        & 3200            \\
  \hline
  2        & 20          & 11        & 16        & 3200            \\
  \hline
  3        & 20          & 16        & 16        & 3300            \\
  \hline
  4        & 22          & 10        & 16        & 3200            \\
  \hline
  5        & 16          & 13        & 17        & 3300            \\
  \hline
  6        & 22          & 16        & 16        & 3400            \\
  \hline
\end{tabular}

Tính hệ số co giãn của cầu theo giá., hệ số co giãn của cầu theo thu nhập, hệ số co giãn
chéo của cầu hàng X theo giá hàng Y. Cho biết X là hàng gì? X và Y có mối quan hệ gì?

\begin{enumerate}
  \item Tính hệ số co giãn của cầu theo giá.,
        \\
        ta có công thức hệ số co giãn của cầu theo giá.
        \[ E_{DP} = \frac{\%\Delta Q_D}{\%\Delta P} \]
        với \\
        $\%\Delta Q_D$: phần trăm thay đổi của lượng cầu \\
        $\%\Delta P$: phần trăm thay đổi của giá \\

        ở đây chúng ta không có phương trình đường cầu \\
        nếu chúng ta xấp xỉ phương trình đường cầu theo phương pháp
        La-grăng thì có lẽ bài toán vượt quá trình độ toán học của môn học này
        vì nó là 1 phần của môn giải tich số, môn này không dạy cho khoa kinh tế trường
        chúng ta \\
        chúng ta lưu ý định nghia trong giáo trình slide 4 tuần 4 \\
        \textbf{Phương pháp trung điểm: tính phần trăm thay đổi theo
          cách chia mức thay đổi cho giá trị trung bình của điểm
          đầu và điểm cuối}\\

        chúng ta lấy điểm đầu là khảo sát số 1 và điểm cuối là khảo sát số 6
        và áp dụng công thức
        \[ E_{DP} = \frac{\%\Delta Q_D}{\%\Delta P} =
          \frac{\frac{\Delta Q_D}{Q_D} \times 100 \% }{ \frac{\Delta P}{P} \times 100 \%  } =
          \frac{\Delta Q_D}{\Delta P}  \times
          \frac{\frac{P_1 + P_2}{2}}{\frac{Q_{D_1} + Q_{D_2}}{2}} \]
        với

        $\Delta Q_D = Q_1 - Q_6 = 20 - 22 = -2$

        $\Delta P = P_1 - P_6 = 15 - 16 = -1$

        $ P = \frac{15 + 16}{2} = 15.5$

        $ Q_D = \frac{20 + 22}{2} = 21$

        \[ E_{DP} = \frac{-2}{-1} * \frac{15.5}{24} = \frac{31}{21} = 1.476  \]


  \item  hệ số co giãn của cầu theo thu nhập,

        tương tự câu trên, chúng ta lấy điểm đầu là khảo sát 1 và điểm cuổi là khảo sát 6

        chúng ta nhắc lại công thức tính hệ số co giãn của cầu theo thu nhập

        \[ E_{DI} =
          \frac{\% \Delta Q_D}{\% \Delta I}
          = \frac{\frac{\Delta Q_D}{Q_D} * 100 \% }{\frac{\Delta I}{I} * 100 \% }
          = \frac{\Delta Q_D}{\Delta I} * \frac{I}{Q_D}
        \]
        trong đó

        $\% \Delta Q_D$ là phần trăm thay đổi của lượng cầu

        $\% \Delta I$ là phần trăm thay đổi của thu nhập

        $ I = \frac{I_1 + I_2}{2}$

        $ Q_D = \frac{Q_{D_1} + Q_{D_2}}{2}$


        ta sẽ tính như sau

        $ I = \frac{3200 + 3400}{2} = 3300$

        $ Q_D = \frac{20 + 22}{2} = 21$

        $\Delta Q_D = 20 - 22 = -2$

        $\Delta I = 3200 - 3400 = -200$

        kết quả như sau

        \[ E_{DI} = \frac{-2}{-200} * \frac{3300}{21} =  1.571 \]


  \item hệ số co giãn chéo của cầu hàng X theo giá hàng Y.

        tương tự câu trên, chúng ta lấy điểm đầu là khảo sát 1 và điểm cuổi là khảo sát 6

        chúng ta sử dụng công thức tính hệ số co giãn chéo trên khoảng

        \[ E_{DC} = \frac{\% \Delta Q_{DX}}{\% \Delta P_{Y}}
          =  \frac{ \Delta Q_{DX}}{\Delta P_{Y}} * \frac{P_{Y}}{Q_{DX}} \]

        với

        $P_{Y} = \frac{P_{Y_1} + P_{Y_2}}{2}$

        $Q_{DX} = \frac{Q_{DX_1} + Q_{DX_2}}{2}$

        thay vào công thức tính như sau

        $P_{Y} = \frac{15 + 16 }{2}$

        $Q_{DX} = \frac{20 + 22}{2}$

        $\Delta Q_{DX} = 20 -22 = -2$

        $\Delta P_{Y} = 15 - 16 = -1$

        \[ E_{DC} = \frac{ \Delta Q_{DX}}{\Delta P_{Y}} * \frac{P_{Y}}{Q_{DX}}
          = \frac{-2}{-1} * \frac{31}{42} = \frac{31}{21} =  1.476 \]


  \item Cho biết X là hàng gì? X và Y có mối quan hệ gì?

        ta có $E_{DI} = 1.571$ là 1 số dương, do đó X là hàng thông thường theo slide 8 tuần 4 \\
        ngoài ra X còn là hàng xa xỉ vì nó lớn hơn 1

        ta có $E_{DC} = 1.476$ là 1 số dương do vậy X và Y là 2 loại hàng hóa thay thế theo slide 9 tuần 4 \\


        \section{Phương trình đường cầu cà phê được cho bởi: $Q_X = 15 - 3P_X + 0,08I - 0,6P_Y$}

        $Q_X$ là lượng cầu cà phê (nghìn tấn); PX là giá cà phê (\$/kg); I là thu nhập của người tiêu
        dùng (nghìn \$/năm); PY là giá đường (\$/kg)

        \begin{enumerate}[a.]
          \item Giả sử rằng hiện nay I = 25, $P_Y$ = 5.  Tính hệ số co giãn của cầu cà phê theo giá cà phê
                khi giá cà phê $P_X$ = 1 và khi giá cà phê tăng từ 1 lên 3 theo phương pháp trung điểm. \\ \\
                viết lại phương trình  $Q_X = 15 - 3P_X + 0,08 * 25 - 0,6 * 5$ \\
                $Q_X = 15 - 3P_X + 2 - 3 = 14 - 3P_X$

                Tính hệ số co giãn của cầu tại mức giá bẳng 1;
                ta áp dụng công thức
                \[ E_{DP} = \frac{\%\Delta Q_D}{\%\Delta P} =
                  \frac{\Delta Q_D}{\Delta P} \times \frac{P}{Q_D} = Q_D' \times \frac{P}{Q_D} \]

                $Q_X' = ( 14 - 3P_X)' = -3$

                \[ E_{DP} = \frac{\%\Delta Q_D}{\%\Delta P} = -3 \times \frac{1}{14 - 3 * 1}
                  = -3 \times \frac{1}{15} = -0.2 \]

                và khi giá tăng từ 1 lên 3 theo phương  pháp trung điểm.

                \[ E_{DP} = \frac{\%\Delta Q_D}{\%\Delta P} =
                  \frac{\Delta Q_D}{\Delta P}  \times
                  \frac{\frac{P_1 + P_2}{2}}{\frac{Q_{D_1} + Q_{D_2}}{2}}
                \]
                nhắc lại phương trình cầu $Q_X = 14 - 3P_X$

                $Q_1 = 14 - 3 * 1 = 11$

                $Q_3 = 14 - 3 * 3 = 5$

                $\Delta Q_D = Q_1 - Q_3 = 11 - 5 = 6$

                $\Delta P = 1 - 3 = -2$

                \[ E_{DP} = \frac{\%\Delta Q_D}{\%\Delta P} =
                  \frac{6}{-2}  \times
                  \frac{\frac{1 + 3}{2}}{\frac{11 + 5}{2}} =
                  -3  \times
                  \frac{4}{16} = \frac{-3}{4} = -0.75
                \]

          \item Giả sử rằng $P_X$ = 2, $P_Y$ = 5. Tính hệ số co giãn của cầu theo thu nhập khi thu nhập I =  50 và khi thu nhập tăng từ 50 lên 100.

                viết lại phương trình  $Q_X = 15 - 3 * 2 + 0,08I - 0,6 * 5$ \\
                $Q_X = 15 - 6 + 0.08I - 3 = 6 + 0.08I$

                chúng ta nhắc lại công thức tính hệ số co giãn của cầu theo thu nhập từ 50 lên 100

                \[ E_{DI} =
                  \frac{\% \Delta Q_D}{\% \Delta I}
                  = \frac{\frac{\Delta Q_D}{Q_D} * 100 \% }{\frac{\Delta I}{I} * 100 \% }
                  = \frac{\Delta Q_D}{\Delta I} * \frac{I}{Q_D}
                \]
                trong đó

                $\% \Delta Q_D$ là phần trăm thay đổi của lượng cầu

                $\% \Delta I$ là phần trăm thay đổi của thu nhập

                $ I = \frac{I_1 + I_2}{2}$

                $ Q_D = \frac{Q_{D_1} + Q_{D_2}}{2}$


                ta sẽ tính như sau

                $ I = \frac{50 + 100}{2} = 75$

                $ Q_D = \frac{6 + 0.08 * 50  + 6 + 0.08 * 100}{2} = 12$

                $\Delta Q_D = 6 + 0.08 * 50  - 6 - 0.08 * 100 = -4$

                $\Delta I = 50 - 100= -50$

                kết quả như sau

                \[ E_{DI} = \frac{-4}{-50} * \frac{75}{12} = \frac{2}{25} * \frac{25}{4} = 0.5  \]

                ta có công thức tính $E_{DI}$ theo điểm như sau
                \[ E_{DI} =
                  \frac{\% \Delta Q_D}{\% \Delta I}
                  = \frac{\frac{\Delta Q_D}{Q_D} * 100 \% }{\frac{\Delta I}{I} * 100 \% }
                  = Q_D' * \frac{I}{Q_D}
                \]

                ta có $Q_X = 15 - 6 + 0.08I - 3 = 6 + 0.08I$

                nên  $Q_X' = 0.08$

                $ I = 50$

                $ Q_D = 6 + 0.08 * 50 = 10$

                \[ E_{DI} =
                  Q_D' * \frac{I}{Q_D}
                  = 0.08 * \frac{50}{10}
                  = 0.4
                \]


          \item Giả sử rằng $P_X$ = 2, I = 25. Tính hệ số co giãn chéo của cầu cà phê theo giá đường khi
                giá đường $P_Y$ = 6 và khi giá đường tăng từ 2 lên 6 theo phương pháp co giãn khoảng.

                viết lại phương trình  $Q_X = 15 - 3P_X + 0,08I - 0,6P_Y$ \\
                $Q_X = 15 - 6 + 2 - 0,6P_Y = 11 - 0,6P_Y$

                chúng ta sử dụng công thức tính hệ số co giãn chéo trên khoảng

                \[ E_{DC} = \frac{\% \Delta Q_{DX}}{\% \Delta P_{Y}}
                  =  \frac{ \Delta Q_{DX}}{\Delta P_{Y}} * \frac{P_{Y}}{Q_{DX}} \]

                với

                $P_{Y} = \frac{P_{Y_1} + P_{Y_2}}{2}$

                $Q_{DX} = \frac{Q_{DX_1} + Q_{DX_2}}{2}$

                từ 2 lên 6
                thay vào công thức tính như sau

                $P_{Y} = \frac{2 +6 }{2}$

                $Q_{DX} = \frac{11 - 0.6 * 2 + 11 - 0.6 * 6}{2}$

                $\Delta Q_{DX} = 2 - 6 = -4$

                $\Delta P_{Y} = 11 - 0.6 * 2 - 11 + 0.6 * 6 = 2.4$

                \[ E_{DC} = \frac{ \Delta Q_{DX}}{\Delta P_{Y}} * \frac{P_{Y}}{Q_{DX}}
                  = \frac{-4}{2.4} * \frac{8}{22 - 4.8} = 0.775 \]


                áp dụng công thức hệ số co giãn theo điểm tại $P_Y$ = 6

                \[ E_{DC} = \frac{\% \Delta Q_{DX}}{\% \Delta P_{Y}}
                  =  \frac{ \Delta Q_{DX}}{\Delta P_{Y}} * \frac{P_{Y}}{Q_{DX}}
                  = Q_D' *  \frac{P_{Y}}{Q_{DX}} \]

                ta có

                \[ E_{DC} = Q_D' *  \frac{P_{Y}}{Q_{DX}}
                  = -0.6 *  \frac{6}{11 - 0.6 * 6}
                  = -0.6 * \frac{6}{7.4} = -0.486 \]
        \end{enumerate}


\end{enumerate}

\section{ Một công ty sản xuất thép có: hệ số co giãn của cầu về thép đối với giá thép}

Một công ty sản xuất thép có: hệ số co giãn của cầu về thép đối với giá thép là -2, hệ
số co giản của cầu về thép đối với thu nhập là 1,5; hệ số co giãn của cầu về thép theo giá của
nhôm là 0,5. Lượng bán thép năm nay của công ty là 1000 tấn. Công ty dự báo trong năm tới
giá của thép tăng 6\%, thu nhập của người tiêu dùng tăng 4\% và giá của nhôm giảm 4\%. Tổng
ảnh hưởng của các yếu tố trên làm Lượng bán thép của công ty trong năm tới sẽ thay đổi
như thế nào? Và Dự tính lượng bán thép của công ty trong năm tới là bao nhiêu?
\\


theo như đầu bài chúng ta có
$E_{DP} = -2$,  $E_{DI} = 1.5$ $E_{DC} = 0.5$ $Q = 1000$ \\
$\% \Delta P = 6$, $\% \Delta I = 4$, $\% \Delta P_Y = -4$
\\
$E_{DC} = 0.5$ lớn hơn 0 nên nhôm và thép là 2 hàng hóa thay thế \\
$E_{DP} = -2$ nên độ lớn của trị tuyệt đối lớn hơn 1, nên lượng cầu co giãn theo giá \\
$E_{DI} = 1.5$ lớn hơn 1 nên cầu co giãn theo thu nhập và thép là hàng hóa thông thường


ta có các công thức

\[ E_{DP} = \frac{\%\Delta Q_D}{\%\Delta P}\]

năm mới $\%\Delta P = 6$ do vậy ta có $-2 = \frac{\%\Delta Q_D}{6}$ \\
vậy $\%\Delta Q_D = -12$ do đó lượng cầu theo giá của năm mới sẽ là \\
$Q_D = 1000 - 0.12 * 1000 = 880$

\[ E_{DI} =
  \frac{\% \Delta Q_D}{\% \Delta I} \]

tương tự ta tính như sau \\
năm mới $\%\Delta I = 4$ do vậy ta có $1.5 = \frac{\%\Delta Q_D}{4}$ \\
vậy $\%\Delta Q_D = 6$ do đó lượng cầu theo thu nhập của năm mới sẽ là \\
$Q_D = 1000 + 0.06 * 1000 = 1060$


\[ E_{DC} = \frac{\% \Delta Q_{DX}}{\% \Delta P_{Y}}
  =  \frac{ \Delta Q_{DX}}{\Delta P_{Y}} * \frac{P_{Y}}{Q_{DX}}
  = Q_D' *  \frac{P_{Y}}{Q_{DX}} \]

năm mới $\%\Delta P_{Y} = -4$ do vậy ta có $0.5 = \frac{\%\Delta Q_D}{-4}$ \\
vậy $\%\Delta Q_D = -2$ do đó lượng cầu theo thu nhập của năm mới sẽ là \\
$Q_D = 1000 - 0.02 * 1000 = 980$

Tổng ánh hưởng sẽ là trung bình của cả 3 yếu tố:
(880 + 1060 + 980) / 3 = 973

do đó có thể thấy lượng thép bán ra sẽ bị giảm
