\chapter{CÁC LỰC LƯỢNG 
CUNG CẦU TRÊN THỊ 
TRƯỜNG}

\section{Điều gì xảy ra với giá và lượng cân bằng trên thị trường máy lạnh trong các
  tình huống sau:}

\begin{enumerate}[(a)]
    \item Thời tiết trở lên nóng bất thường, người bán không thay đổi lượng bán ra.
    \item  Lượng máy lạnh nhập khẩu gia tăng
    \item  Giá điện tăng cao, người bán không thay đổi lượng bán ra.
    \item  Các nhà khoa học khuyến cáo, máy lạnh có hại cho sức khỏe.
    \item  Thu nhập của người tiêu dùng giảm mạnh do suy thoái kinh tế.
    \item  Nhiều doanh nghiệp rời bỏ thị trường do chính phủ tăng thuế.
    \item  a và b xảy ra đồng thời nhưng ảnh hưởng của a mạnh hơn.
    \item  e và f xảy ra đồng thời
\end{enumerate}

\section{Cung – cầu về sản phẩm Y có dạng: $Q_S = 2P - 8$ và $Q_D = 15 - 0.5P$
  (trong đó Q tính bằng triệu tấn, P tính bằng nghìn đồng/tấn)}

\begin{enumerate}[(a)]
    \item Xác định giá và sản lượng cân bằng của sản phẩm Y.
    \\
    $Q_S = Q_D$
    \\ 
    $2P - 8 = 15 - 0.5P$
    \\
    $2.5P = 15 + 8 $
    \\
    $2.5P = 23$
    \\ 
    $P = 9.2$
    \\
    $Q_S = Q_D = 10.4$

    \begin{tikzpicture}
        \draw [->] (0,0) -- (5,0)node[right] {$Q$};
        \draw [->] (0,0) -- (0,5)node[above] {$P$};
        \draw[scale = 0.1, domain=0:20, smooth, variable=\x, color=blue] plot ({\x}, {\x /2 + 4});
        \draw[scale = 0.1, domain=0:20, smooth, variable=\x, color=red] plot ({\x}, {30 - 2 * \x});
        \filldraw[scale = 0.1, color=purple] (10.4, 9.2) circle (5pt) node[anchor=west]{A (10.4, 9.2)};
    \end{tikzpicture}

    \item Vì một lý do nào đó lượng cầu giảm 1 triệu tấn ở mọi mức giá, khi đó giá và lượng thay
          đổi như thế nào. Vẽ đồ thị minh họa câu a và câu b trên cùng một đồ thị
    \\ giá giảm , lượng cũng giảm
    \\
    cũ $Q_D = 15 - 0.5P$ 
    \\
    mới $Q_D = 14 - 0.5P$
    \\
    \begin{tikzpicture}
        \draw [->] (0,0) -- (5,0)node[right] {$Q$};
        \draw [->] (0,0) -- (0,5)node[above] {$P$};

        \draw[scale = 0.2, domain=0:20, smooth, variable=\x, color=green] plot ({\x}, {28 - 2 * \x});

        \draw[scale = 0.2, domain=0:20, smooth, variable=\x, color=blue] plot ({\x}, {\x /2 + 4});
        \draw[scale = 0.2, domain=0:20, smooth, variable=\x, color=red] plot ({\x}, {30 - 2 * \x});
        \filldraw[scale = 0.2, color=purple] (10.4, 9.2) circle (5pt) node[anchor=west]{A};
    \end{tikzpicture}
    
    \item Do giá nguyên liệu sản xuất sản phẩm Y giảm nên lượng cung tăng 10 \% tại mọi mức giá. Xác định giá và lượng cân bằng mới. Vẽ đồ thị minh họa câu a và câu c trên cùng  một đồ thị
    \\
    phương trình cũ :
    $Q_S = 2P - 8$ 

    trước đấy với số tiền 2P - 8
    chúng ta mua được $Q_S$
    do cung tăng 10 \% với mọi mức gía
    ý ở ở đây là P giữ nguyên
    \\
    thì ta sẽ mua được như sau:
    2P - 8  + 0.1 * (2P - 8)

    $Q_S = (2P - 8) + 0.1 * (2P - 8)$

    $Q_S = 2.2P - 8.8$
   
    ta tìm điểm cân băng mới

    $2.2P - 8.8 = 15 - 0.5P$

    $ 2.7P = 15 + 8.8 = 23.8 $

    $ P_{cb} = 8.81$

    $Q_{cb} = 15 - 0.5 * 8.81 =  10.59$
    
    $Q_S = 2.2P - 8.8$

    $2.2 P = Q_S + 8.8$

    $P = Q_S / 2.2 + 4$



    \begin{tikzpicture}
        \draw [->] (-3,0) -- (5,0)node[right] {$Q$};
        \draw [->] (0,0) -- (0,5)node[above] {$P$};
        \draw[scale = 0.2, domain=-10:20, smooth, variable=\x, color=green] plot ({\x}, {  \x / 2.2 + 4});
        
        \draw[scale = 0.2, domain=-10:20, smooth, variable=\x, color=blue] plot ({\x}, { 0.5 * \x  + 4});

        \filldraw[scale = 0.2, color=purple] (10.59, 8.81) circle (5pt) node[anchor=west]{A};
        
        \draw[scale = 0.2, domain=0:20, smooth, variable=\x, color=red] plot ({\x}, {30 - 2 * \x});

       
    \end{tikzpicture}

    \item Khi giá bán trên thị trường là 8 nghìn đồng/tấn thì thị trường xảy ra tình trạng gì? doanh
          thu thu được tại mức giá này là bao nhiêu?

        thiếu hụt hàng hóa 
        doanh thu tính như sau $Q_D = 15 - 0.5P$ $Q_D = 15 - 0.5 * 8$ $Q_D = 11$

        doanh thu bằng Q * P = 11 * 8 = 88

        \begin{tikzpicture}
            \draw [->] (0,0) -- (5,0)node[right] {$Q$};
            \draw [->] (0,0) -- (0,5)node[above] {$P$};
            \draw[scale = 0.2, domain=0:20, smooth, variable=\x, color=blue] plot ({\x}, {\x /2 + 4});
            \draw[scale = 0.2, domain=0:20, smooth, variable=\x, color=red] plot ({\x}, {30 - 2 * \x});
            \filldraw[scale = 0.2, color=purple] (10.4, 9.2) circle (5pt) node[anchor=west]{A};

            \draw[scale = 0.2, domain=0:20, smooth, variable=\x, color=green] (8,8) -- (11, 8);
            
            \filldraw[scale = 0.2] (8,8) circle (5pt) node[anchor=north]{B};
        \end{tikzpicture}
    \item Khi giá bán trên thị trường là 11 nghìn đồng/tấn thì thị trường xảy ra hiện tượng dư cung
          hay dư cầu? Tính mức dư cung hoặc dư cầu? Tính doanh thu thu được tại mức giá này là
          bao nhiêu?

          dư thừa hàng hóa

          doanh thu tính như sau $Q_S = 2P - 8$ $Q_S = 2 * 11 - 8$ $Q_S = 14$

           doanh thu bằng Q * P = 11 * 14 = 154
           
          \begin{tikzpicture}
            \draw [->] (0,0) -- (5,0)node[right] {$Q$};
            \draw [->] (0,0) -- (0,5)node[above] {$P$};
            \draw[scale = 0.2, domain=0:20, smooth, variable=\x, color=blue] plot ({\x}, {\x /2 + 4});
            \draw[scale = 0.2, domain=0:20, smooth, variable=\x, color=red] plot ({\x}, {30 - 2 * \x});
            \filldraw[scale = 0.2, color=purple] (10.4, 9.2) circle (5pt) node[anchor=west]{A};

            \draw[scale = 0.2, domain=0:20, smooth, variable=\x, color=green] (14,11) -- (9.5, 11);

            \filldraw[scale = 0.2] (9.5, 11) circle (5pt) node[anchor=east]{B};

        \end{tikzpicture}
\end{enumerate}



\section{ Cho số liệu về cung – cầu sản phẩm A như sau:}
\begin{tabular}{|c|c|c|}
    Giá (100đ/ 1kg) & Lượng cầu (kg) & Lượng cung(kg) \\
    \hline
    7               & 20             & 11             \\
    \hline
    8               & 19             & 13             \\
    \hline
    9               & 18             & 15             \\
    \hline
\end{tabular}


\begin{enumerate}[(a)]
    \item Viết phương trình đường cung, đường cầu, xác định giá và lượng cân bằng. Doanh thu
          tại trạng thái cân bằng.
          \\
          chúng ta nhắc lại về phương pháp tính phương trình đường thẳng trong hệ tọa độ Đề Các

          phương trình đường thẳng
          đi qua 2 điểm trong hệ tọa độ Đề Các

          ta có trục Ox và trục Oy

          $A * (x - x_0) + B * (y - y_0) = 0$

          ta đã có $x_0$ và $y_0$
          vi dụ $x_0 = 20$ và $y_0 = 7$

          ta cần tìm A và B
          chúng ta nhơ lại rằng (A, B) là véc tơ pháp tuyến  của đương thẳng đi qua 2 điểm cho trước

          muốn tìm  vec tơ pháp tuyến ta cần tìm véc tơ chỉ phương

          vec tơ chỉ phương sẽ tính như sau

          giả sử chúng tâ có 2 điểm
          $M(20, 7)$ $N(19, 8)$

          véc tơ MN = (19 - 20, 8 - 7) = (-1, 1)

          vậy ta đã có véc tơ chỉ phương

          vec tơ pháp tuyến tính như sau

          công thức
          chỉ phương = (C, D)
          pháp tuyến = (-D, C)

          MN =  (-1, 1)
          $\Rightarrow$ pháp tuyến = (-1, -1)

          phương trình đường cầu

          $A * (x - x_0) + B * (y - y_0) = 0$

          (A, B) = (-1-, -1)

          $x_0 = 20$ $y_0= 7$

          $-1 * (x - 20) + (-1) * (y - 7) = 0$

          $-x + 20 - y + 7 = 0$

          $-x - y + 27 = 0$

          $x = 27 - y$

          $Q_D = 27 - P$

          đường cung
          MN = (13 - 11, 8 - 7) = (2, 1)
          pháp tuyến = (-1, 2)

          phương trình đường cung

          $-1 * (x - 11) + (2) * (y - 7) = 0$

          $-x + 11 + 2y - 7 = 0$

          $-x + 2y + 4 = 0$

          $4 + 2y = x$

          $Q_S = 4 + 2P$



          kết luận
          ta có

          $Q_S = 2P + 4$,
          $Q_D = 27 - P$

          $P = Q_S / 2 - 2$ $y = x / 2 - 2$

          $P = 27 - Q_D$

          tính điểm giao của 2 đường thăng - điểm cân bằng

          $Q_S = Q_D$

          $2P + 4 = 27 - P$

          $3P = 23 \rightarrow P = 7.6, Q = 19.4$

          Doanh thu tại trạng thái cân bằng.:
          P * Q = 7.6 * 19.4

          \begin{tikzpicture}
              \draw [->] (0,0) -- (5,0)node[right] {$Q$};
              \draw [->] (0,0) -- (0,5)node[above] {$P$};
              \draw[scale = 0.1, domain=0:30, smooth, variable=\x, color=blue] plot ({\x}, {\x /2 - 2});
              \draw[scale = 0.1, domain=0:30, smooth, variable=\x, color=red] plot ({\x}, {27 - \x});
              \filldraw[scale = 0.1, color=purple] (19.4, 7.6) circle (5pt) node[anchor=west]{A};
          \end{tikzpicture}


    \item Vì lý do nào đó, lượng cung sản phẩm A tăng lên một lượng là 6 kg ở mỗi mức giá. Hãy
          xác định mức giá và sản lượng, tổng doanh thu tại trạng thái cân bằng mới?.

          cũ
          $Q_S = 2P + 4$,
          $Q_D = 27 - P$

          mới
          $Q_S = (2P + 4) + 6$,
          $Q_D = 27 - P$

          $ \Rightarrow Q_S = 2P + 10$,
          $Q_D = 27 - P$

          $Q_S = Q_D$

          $2P + 10 = 27 - P$

          $3P = 17 \rightarrow P = 6.3, Q = 21.7 \rightarrow$ tổng doanh thu là 6.3 * 21.7

    \item Giả sử Chính phủ áp đặt giá bán trên thị trường là 11 nghìn đồng/kg và hứa mua hết phần
          sản phẩm thừa, thì số tiền chính phủ phải chi ra là bao nhiêu?

    đường thẳng song song với trục hoành
    y = 11 là đường giá cố định của chính phủ

    ta cần tính giáo của đường cung với đường áp giá để tìm ra lượng hàng cần tiêu thụ

    $Q_S = 2P + 4$,
    thay P  = 11 vào ta có 

    $Q_S = 2 * 11 + 4 = 26$,

    $Q_D = 27 - P$
    thay P  = 11 vào ta có 

    $Q_D = 27 - 11 = 16$

    lượng Dư thừa = $Q_S - Q_D = 26 - 16 = 10$

    vậy chính phủ cần mua 10 kg
    , số tiền bỏ ra là 10 * 11 = 110 

          \begin{tikzpicture}
            \draw [->] (0,0) -- (5,0)node[right] {$Q$};
            \draw [->] (0,0) -- (0,5)node[above] {$P$};
            \draw[scale = 0.1, domain=0:30, smooth, variable=\x, color=blue] plot ({\x}, {\x /2 - 2});
            \draw[scale = 0.1, domain=0:30, smooth, variable=\x, color=red] plot ({\x}, {27 - \x});
            \filldraw[scale = 0.1, color=purple] (19.4, 7.6) circle (5pt) node[anchor=west]{A};

            \draw[scale = 0.1, domain=0:30, smooth, variable=\y, color=green] plot ({\y}, {11});

            \filldraw[scale = 0.1, color=purple] (26, 11) circle (5pt) node[anchor=south]{S};

            \filldraw[scale = 0.1, color=purple] (16, 11) circle (5pt) node[anchor=south]{D};
        \end{tikzpicture}
\end{enumerate}


\section{ Cho thị trường hàng hóa A có phương trình đường cung và đường cầu như 
sau: $P_S = 0,2Q - 10$ và $P_D = 20 - 0.2Q$ (bỏ qua đơn vị của giá và lượng)}

\begin{enumerate}[a.]
    \item Xác định Giá và sản lượng cân bằng của thị trường?
    \\
    $P_S = 0,2Q - 10$ và $P_D = 20 - 0.2Q$

    $0,2Q - 10 = 20 - 0.2Q$

    $0.4Q = 30$

    $Q = 75$

    $P = 20 - 0.2 * 75 = 5$

    \item Giả sử giá bán trên thị trường là P = 10 thì thị trường xảy ra tình trạng gì? Doanh 
    thu thu được tại mức giá này bằng bao nhiêu?
   

    \begin{tikzpicture}
        \draw [->] (0,0) -- (8,0)node[right] {$Q$};
        \draw [->] (0,0) -- (0,3)node[above] {$P$};
        \draw[scale = 0.07, domain=0:120, smooth, variable=\x, color=blue] plot ({\x}, {\x /5 - 10});
        \draw[scale = 0.07, domain=0:80, smooth, variable=\x, color=red] plot ({\x}, {20 - 0.2 * \x});
        \filldraw[scale = 0.07, color=black] (75, 5) circle (5pt) node[anchor=south]{CB};

        \draw[scale = 0.07, domain=0:120, smooth, variable=\y, color=green] plot ({\y}, {10});

        \filldraw[scale = 0.07, color=black] (50, 10) circle (5pt) node[anchor=south]{Z};

    \end{tikzpicture}

    $P = 10  > P_{CB}= 5$
     
    nên  cầu giảm , cung dư, tức là dư thừa hàng hóa

    $P_D = 20 - 0.2Q \Rightarrow P_D - 20 = -0.2 Q $

    $P_D = 10$

    $10 - 20 = -0.2Q  \Rightarrow Q = 50$

    Doanh thu = P * Q = 10 * 50 = 500

    \item Do nhiều hàng hóa thay thế cho hàng hóa A xuất hiện nên lượng cầu về hàng hóa A 
    giảm 20\% tại mọi mức giá. Hãy tính tác động của của việc giảm cầu này đối với giá ?

    \begin{tikzpicture}
        \draw [->] (0,0) -- (8,0)node[right] {$Q$};
        \draw [->] (0,0) -- (0,3)node[above] {$P$};
        \draw[scale = 0.07, domain=0:120, smooth, variable=\x, color=blue] plot ({\x}, {\x /5 - 10});
        \draw[scale = 0.07, domain=0:80, smooth, variable=\x, color=red] plot ({\x}, {20 - 0.2 * \x});
        \filldraw[scale = 0.07, color=black] (75, 5) circle (5pt) node[anchor=south]{CB};
        
        \draw[scale = 0.07, domain=0:80, smooth, variable=\x, color=green] plot ({\x}, { 10 - \x / 4 });

    \end{tikzpicture}

    $P_D = 20 - 0.2Q$

    $ -P_D  + 10 = 0.2Q$

    cũ $Q = 50 - 5P_D$

    mới $Q_D = 0.8 * (50 - 5P_D) \Rightarrow Q_D = 40 - 4P_D$

    $P_D = 10 - 0.25Q_D$

    ta tìm điểm cân bằng mới 

    $P_S = 0,2Q - 10 = 10 - 0.25Q$

    $0.45Q = 20 \Rightarrow Q = 44.4$

    $P = 0,2Q - 10 = -1.12$

    từ đây ta có thể thấy là giá thành của sản phẩm A rơi xuống dưới 0, và nhà sản xuất phải đưa thêm tiền cho khách hầng để bán sản phẩm , ở mức cân bằng cảu thị trường việc đó đã từng xảy ra với giá dầu khi dịch covid xảy ra vào năm ngoái 

    \item  Do giá hàng B là hàng thay thế cho A giảm nên lượng cầu về A giảm một lượng 
    tuyệt đối tại mọi mức giá. Biết lượng cân bằng mới bây giờ là 60. Lập phương trình 
    đường cầu mới?

    từ lượng cân bằng là Q = 60 và $P_S = 0,2Q - 10$ là cố định
     ta chỉ ra P = 0.2 * 60 - 10 = 2

     lưu ý rằng lượng cầu giảm tuyệt đối với mọi mức Giá
     tức là đường cầu mới sẽ song song với đường cầu cũ 

     chúng ta có thể giải thích việc này qua phương trình 

     $y = ax + b$ khi b thay đổi thì đường thẳng mới song song với đường thẳng cũ đó là ý của chữ giảm "tăng"  tuyệt đối với mọi mức Giá

     vậy công việc là viết phương trình đường mới với hệ số cũ 
     và đi qua điểm cân bằng mới 

     cụ thể phương trình cũ là 
     $P_D = 20 - 0.2Q$ viết lại là 
     $P_D + 0.2Q - 20 = 0$, 
     tá có vec tơ pháp tuyến ở đây là (0.2, 1)
     phương trình này đi qua  điểm (60, 2)

     phương trình mới sẽ là 
     $0.2 (x - 60) + 1 (y - 2) = 0$

     $0.2x - 12 + y - 2 = 0 $

     $0.2x + y - 14 = 0$

     $P_D = 14 - 0.2 Q$

     \begin{tikzpicture}
        \draw [->] (0,0) -- (8,0)node[right] {$Q$};
        \draw [->] (0,0) -- (0,3)node[above] {$P$};
        \draw[scale = 0.07, domain=0:120, smooth, variable=\x, color=blue] plot ({\x}, {\x /5 - 10});
        \draw[scale = 0.07, domain=0:80, smooth, variable=\x, color=red] plot ({\x}, {20 - 0.2 * \x});
        \filldraw[scale = 0.07, color=black] (75, 5) circle (5pt) node[anchor=south]{CB};
        
        \draw[scale = 0.07, domain=0:80, smooth, variable=\x, color=green] plot ({\x}, { 14 - \x / 5 });

        
    \end{tikzpicture}
\end{enumerate}

\section{  Hàm cầu về sản phẩm X trên thị trường được cho bởi phương trình: $P = 100 - 0,05Q_D$; trong đó Q là sản lượng tính bằng đơn vị, P tính bằng \$. Cung sản phẩm X 
luôn cố định ở mức 1100 đơn vị.}

\begin{enumerate}[a.]
    \item Tính giá và sản lượng cân bằng của sản phẩm X.
    
    phương trình đường cung có dạng
    x = b tức là song song với trục tung P

    $Q_S = 1100$

    $P = 100 - 0,05Q_D$

    $0.05Q_D = 100 - P$

    viết lại phương trình đường cầu
    $Q_D = 2000 - 20P$

    tại điểm cân bằng 

    $Q_D = Q_S$

    $1100 = 2000 - 20P$

    $20P = 2000 - 1100 = 900$

    $P = 45$

    vậy điểm cân bằng là 
    (1100, 45)
    Q = 1100, P = 45

    \begin{tikzpicture}
        \draw [->] (0,0) -- (15,0)node[right] {$Q$};
        \draw [->] (0,0) -- (0,5)node[above] {$P$};             
        \draw[scale = 0.08, domain=0:150, smooth, variable=\x, color=red] plot ({\x}, {10 - 0.05 * \x});
        \draw [-, color=blue, scale = 0.08] (110,0) -- (110,80);             
        \filldraw[scale = 0.08, color=black] (110, 4.5) circle (5pt) node[anchor=west]{CB};
    \end{tikzpicture}
    

    \item  Giả sử nhờ quảng cáo, lượng cầu tại mỗi mức giá tăng thêm 15\%. Giá và sản lượng 
    cân bằng mới trên thị trường là bao nhiêu. Vẽ hình minh họa?

    phương trình đường cầu
    $Q_D = 2000 - 20P$

    tại mỗi mức giá chúng ta tăng 15\%

    $Q_D = (2000 - 20P) + 0.15( 2000 - 20P)$

    $Q_D = 1.15(2000 - 20P)$

    $Q_D = 2300 - 23P$

    để tìm giá và sản lượng cân bằng mới 
    ta lưu ý rằng lượng cung không đổi và là 1100
    $\Rightarrow$ ta cần tính P

    $1100 = 2300 - 23P$

    $23P = 2300 - 1100 = 1200$

    $P = 52.17$

    điểm cân bằng mới 
    Q = 1100, P = 52.17

    
    $Q_D = 2300 - 23P$

    $P = 100 - Q_D / 23 $

    \begin{tikzpicture}
        \draw [->] (0,0) -- (15,0)node[right] {$Q$};
        \draw [->] (0,0) -- (0,5)node[above] {$P$};             
        \draw[scale = 0.08, domain=0:150, smooth, variable=\x, color=red] plot ({\x}, {10 - 0.05 * \x});
        \draw [-, color=blue, scale = 0.08] (110,0) -- (110,80);         
        
        \draw[scale = 0.08, domain=0:150, smooth, variable=\x, color=green] plot ({\x}, {10 - \x / 23});
        
        \filldraw[scale = 0.08, color=black] (110, 5.217) circle (8pt) node[anchor=west]{CB};
    \end{tikzpicture}

    \item  Khi chính phủ áp đặt giá bán trên thị trường là 50 thì doanh thu là bao nhiêu?
    
    P = 50
    lưu ý rằng P cân bằng là 45 mà lượng cung không thay đổi
    đây gọi là áp giá sàn

    tại điểm P = 50 thì cầu thị trường là $Q_D = 2000 - 20P$

    $Q_D = 2000 - 20 * 50 = 2000  - 1000 = 1000$



    do đó doanh thu 50 * 1000 = 50000
     vì nhà nước không cam kết thu mua sản phẩm thừa 

    \begin{tikzpicture}
        \draw [->] (0,0) -- (15,0)node[right] {$Q$};
        \draw [->] (0,0) -- (0,5)node[above] {$P$};             
        \draw[scale = 0.08, domain=0:150, smooth, variable=\x, color=red] plot ({\x}, {10 - 0.05 * \x});

        \draw[scale = 0.08, domain=0:150, smooth, variable=\y, color=green] plot ({\y}, {5});

        \draw [-, color=blue, scale = 0.08] (110,0) -- (110,80);             
       
        \filldraw[scale = 0.08, color=black] (110, 4.5) circle (5pt) node[anchor=west]{MUA};


    \end{tikzpicture}
    
\end{enumerate}

\section{Bài 6. Xác định hàm cung và hàm cầu trong các trường hợp sau:}

\begin{enumerate}[a.]
    \item Trong một thị trường có 200 người bán và 100 người mua. Những người bán có hàm 
    cung giống nhau là $P = 0,5q + 100$ và những người mua có hàm cầu giống nhau là
    $q = 2250 - 6P$ (trong đó q là nghìn sản phẩm, p là nghìn đồng/sp). Xác định hàm cung, 
    hàm cầu của thị trường.

    chúng ta lưu ý một số định nghĩa sau :
    \begin{itemize}
        \item Cung thị trường bằng tổng cung cá nhân theo chiều ngang. 
        \[ Q_S = \sum_{j=1}^n q_{s_j} \]
        \item Cầu thị trường bằng tổng 
        cầu cá nhân theo chiều  ngang
        \[ Q_D = \sum_{i=1}^n q_{d_i} \]
    \end{itemize}

    ta biến đổi các phương trình dạng giá thành các phương trình dạng lượng 

    $P = 0,5q + 100 \Rightarrow 0.5q = P - 100 \Rightarrow q = 2P - 200$

    $q = 2250 - 6P$ phương trình không cần biến đổi

    ta có 200 người bán vậy phương trình cung thị trường sẽ là như sau

    $Q_S = 200 * q = 200 * (2P - 200) = 400P - 40000$

    ta có 100 người mua vậy phương trình cầu như sau

    $Q_D = 100 * q = 100 * (2250 - 6P) = 225000 - 600P$
    
    \item  Thị trường sản phẩm A có 3 nhóm người tiêu dùng có phương trình đường cầu lần lượt là 
    $P = 20 - 0,001q_A$ ; $q_B = 40.000 - 2.000P$ và $P = 20 - 0,0002q_C$. Và trong thị trường này có 
    250 người bán, mỗi người bán đều có hàm cung giống nhau là $P = 0,1q - 13,6$. Hãy xác định 
    hàm cầu và hàm cung của thị trường sản phẩm A. Xác định giá và lượng cân bằng của thị 
    trường.

    chúng ta biến đổi về phương trình lượng cầu

    $P = 20 - 0,001q_A \Rightarrow 0,001q_A = 20 - P \Rightarrow q_A = 20000 - 1000P$ 
    
    $q_B = 40000 - 2000P$ 
    
    $P = 20 - 0,0002q_C \Rightarrow 0,0002q_C = 20 - P \Rightarrow q_C = 10000 - 5000P$

    vậy phuơng trình đường cầu thị trường là tổng của tất cả các cầu

    $Q_D = 20000 - 1000P + 40000 - 2000P +  10000 - 5000P = 70000 - 8000P$

    biến đổi  $P = 0,1q - 13,6$ về phương trình cung 

    $P = 0,1q - 13,6 \Rightarrow  0,1q = P + 13,6 \Rightarrow q = 10P + 136$

    phương trình tổng cung là 
    $Q_S = 250 * ( 10P + 136) = 2500P + 34000$


    điểm cân bằng 
    $Q_D = Q_S$

    $70000 - 8000P = 2500P + 34000 \Rightarrow 10500P = 36000 \Rightarrow P = 3.42$

    $Q = 2500 * 3.42 + 34000 = 8550 + 34000 = 42550$
    \item  Thị trường của sản phẩm X được mô tả ở đồ thị sau đây:
    Hãy viết phương trình biểu diễn cung, cầu của sản phẩm X

    đường cầu màu xanh lá đi qua điểm (0, 20) và (500, 10)
    ta có vectơ chỉ phương (0 - 500, 20 - 10) = (-500, 10)
    và vectơ pháp tuyến là (-10, -500)
    phương trình sẽ như sau: 
    
    $-10 * (Q_D - 0) + (-500) * (P - 20) = 0$

    $-10Q_D - 500P + 10000 = 0$

    $500P = -10Q_D + 10000$

    $P = 20  -Q_D / 50$

    tương tự với phương trình đường cung đi qua điểm (0, 5)  và (500, 10)
    ta có vectơ chỉ phương (0 - 500, 5 - 10) = (-500, -5)
    và vectơ pháp tuyến là (5, -500)
    phương trình sẽ như sau: 
    
    $5* (Q_S - 0) + (-500) * (P - 5) = 0$

    $5Q_S - 500P + 500 = 0$

    $500P = 5Q_S + 500$

    $P = Q_S / 100 + 1$

\end{enumerate}


