\documentclass{report}
\usepackage{listings}
\usepackage[utf8]{vietnam}
\usepackage{pgfplots}
\usepackage[shortlabels]{enumitem}
\usepackage{hyperref}

\usetikzlibrary{arrows}
\usepgfplotslibrary{polar}
\usepgflibrary{shapes.geometric}
\usetikzlibrary{calc}

\pgfplotsset{compat=1.17}



\begin{document}

\setcounter{chapter}{7}
\chapter{TIẾT KIỆM, ĐẦU TƯ VÀ 
HỆ THỐNG TÀI CHÍNH}

\setcounter{section}{1}
\section{Một nền kinh tế đóng có GDP là 1000 tỷ đồng}
Một nền kinh tế đóng có GDP là 1000 tỷ đồng, thuế là 150 tỷ đồng, tiết kiệm tư 
nhân là 250 tỷ đồng, tiết kiệm chính phủ là -30 tỷ đồng. Hãy tính tiêu dùng của hộ gia 
đình, chi tiêu của chính phủ, tiết kiệm quốc dân và đầu tư.\\
đầu tiên chúng ta ta lưu ý rằng đây là 1 nên kinh tế đóng \\
và có công thức như sau\\
$Y = C + I + G$ với $I = Y - C - G = S_n $ là tiết kiệm quốc dân\\
và trong nền kinh tế đóng Tiết Kiệm = đầu tư\\
Y: GDP. \\
C: Chi cho tiêu dùng cá nhân của các hộ gia đình về hàng hóa và dịch vụ \\
I: Đầu tư phán ánh tổng đầu tư trong nước của khu vực tư nhân \\
G: chi tiêu cho các cấp chính quyền từ trung ương tới địa phương.\\
T:  tổng số tiền thuế mà chính phủ thu được sau 
khi trừ đi các khoản trợ cấp hoặc chuyển giao thu nhập \\
Y - T - C: tiết kiệm tư nhân\\
T - G: tiếp kiệm chính phủ\\
Cán cân ngân sách (B) B = T - G

chúng ta đang có T - G = -30, T = 150, Y - T - C = 250, Y = 1000\\
vậy \\
tiêu dùng của hộ gia đình = C = Y - T -250 = 1000 - 150 - 250 = 600\\
chi tiêu của chính phủ = G = T + 30 = 150 + 30 = 180\\
tiết kiệm quốc dân = đầu tư = Y - C - G = 1000 - 600  - 180 = 280





\end{document}